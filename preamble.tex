\usepackage[top=0.80in, bottom=0.75in, left=0.5in, right=0.5in, footskip=0.3in, bindingoffset=0.25in]{geometry}

\usepackage{graphicx}
\usepackage{amssymb,amsmath}
\usepackage{ifxetex,ifluatex}
\usepackage{fixltx2e} % provides \textsubscript
\usepackage{fancyvrb}
\usepackage{longtable,booktabs}
\usepackage{csquotes}

\let\footruleskip\undefined % Trick to make memoir class work with fancyhdr
\usepackage{fancyhdr}       % http://ctan.org/pkg/fancyhdr\usepackage{fancyhdr}

\usepackage[english]{babel}
\usepackage{lmodern}
\usepackage[T1]{fontenc}
\usepackage[utf8]{inputenc}
\usepackage{textcomp}

\usepackage[unicode=true]{hyperref}
\usepackage[normalem]{ulem}

\usepackage[default]{sourcesanspro}
% Palatino font (ppl must be installed).
%\renewcommand*\rmdefault{ppl}
\usepackage{bera}
\usepackage{inconsolata}
%\renewcommand*\ttdefault{inconsolata} %% Only if the base font of the document is to be typewriter style

%\usepackage{listings}
%\lstnewenvironment{code}{\lstset{language=Haskell,basicstyle=\small\ttfamily}}{}

% use upquote if available, for straight quotes in verbatim environments
\IfFileExists{upquote.sty}{\usepackage{upquote}}{}

% Use microtype if available
% must be loaded after fonts
\IfFileExists{microtype.sty}{%
\usepackage{microtype}
\UseMicrotypeSet[protrusion]{basicmath} % disable protrusion for tt fonts
}{}

% Page styles
% -----------

% Redefinition of the plain style:
\fancypagestyle{plain}{%
\fancyhf{} % clear all header and footer fields
\renewcommand{\headrulewidth}{0pt}
\renewcommand{\footrulewidth}{0.0pt}
\fancyfoot[RO]{\bfseries\large\thepage}
% Display draft info
\ifdraftdoc
\fancyfoot[RE]{\bfseries\textit{Draft \today}}
\fancyfoot[LO]{\bfseries\textit{Draft \today}}
\fi}

\pagestyle{fancy}
\renewcommand{\headrulewidth}{0pt}
\renewcommand{\footrulewidth}{0.0pt}
%\renewcommand{\chaptermark}[1]{\markboth{\MakeUppercase{\chaptername}\space\thechapter:\space#1}{}}
\renewcommand{\chaptermark}[1]{\markboth{#1}{}}
\renewcommand{\sectionmark}[1]{\markright{#1}{}}
\renewcommand{\subsectionmark}[1]{\markright{#1}{}}
\fancyhf{} % clear all header and footer fields
\fancyhead[LE]{\bfseries\large\thepage\space\space\space\space\space\space\space\space\leftmark}
\fancyhead[RE]{\MakeUppercase{\chaptername}\space\thechapter}
\fancyhead[RO]{\bfseries\rightmark\space\space\space\space\space\space\space\space\large\thepage}
% Display draft info
\ifdraftdoc
\fancyfoot[RE]{\bfseries\textit{Draft \today}}
\fancyfoot[LO]{\bfseries\textit{Draft \today}}
\fi

% Footnote
% --------
\renewcommand{\footnoterule}{\noindent\smash{\rule[5pt]{1.0in}{0.1pt}}}

% Table of contents
% -----------------
\hypersetup{linkcolor=black}
\setsecnumdepth{subsection}
\settocdepth{subsection}
\renewcommand{\cftchapterfont}{\bfseries\large}
\renewcommand{\cftsectionfont}{\normalsize\bfseries}
\renewcommand{\cftsubsectionfont}{\normalsize}
\renewcommand{\cftsectiondotsep}{\cftnodots}
\setlength{\cftbeforesectionskip}{0.3em}
\setlength{\cftbeforesubsectionskip}{0.08em}
\setlength{\cftsectionindent}{0em}
\setlength{\cftsubsectionindent}{2.1em}

% Chapter style
% -------------
\makechapterstyle{negerns}{%
\setlength{\beforechapskip}{0pt}
\renewcommand*{\chapnamefont}{\raggedleft\LARGE\scshape}
\renewcommand*{\chapnumfont}{\raggedleft\Huge}
\renewcommand*{\printchapternonum}{%
\vphantom{\printchaptername}%
\vphantom{\chapnumfont 1}%
\afterchapternum}
\renewcommand*{\chaptitlefont}{\raggedleft\huge\bfseries\itshape\fontfamily{ptm}\selectfont}
\renewcommand*{\printchaptertitle}[1]{%
\vskip -\onelineskip
\hrule\vskip\onelineskip \centering\chaptitlefont ##1\\[3ex]}
}

% Section styles
% --------------
\setsecheadstyle{\LARGE\bfseries\itshape\fontfamily{ptm}\selectfont}
\setsubsecheadstyle{\Large\bfseries\itshape\fontfamily{ptm}\selectfont}
\setsubsubsecheadstyle{\large\bfseries\itshape\fontfamily{ptm}\selectfont}

%\renewcommand*{\quotation}{\fontfamily{ptm}\selectfont\itshape}
%\renewcommand*{\quote}{\quotation}{\endquotation}
