%\usepackage[top=0.80in, bottom=0.75in, left=0.5in, right=0.5in, footskip=0.3in, bindingoffset=0.25in]{geometry}

\usepackage{graphicx}
\usepackage{amssymb,amsmath}
\usepackage{ifxetex,ifluatex}
\usepackage{fixltx2e} % provides \textsubscript
\usepackage{fancyvrb}
\usepackage{longtable,booktabs}
\usepackage{csquotes}

\let\footruleskip\undefined % Trick to make memoir class work with fancyhdr
\usepackage{fancyhdr}       % http://ctan.org/pkg/fancyhdr\usepackage{fancyhdr}

\usepackage[english]{babel}
\usepackage{lmodern}
\usepackage[T1]{fontenc}
\usepackage[utf8]{inputenc}
\usepackage{textcomp}

\usepackage[unicode=true]{hyperref}
\usepackage[normalem]{ulem}

\usepackage[default]{sourcesanspro}
% Palatino font (ppl must be installed).
%\renewcommand*\rmdefault{ppl}
\usepackage{bera}
\usepackage{inconsolata}
%\renewcommand*\ttdefault{inconsolata} %% Only if the base font of the document is to be typewriter style

%\usepackage{listings}
%\lstnewenvironment{code}{\lstset{language=Haskell,basicstyle=\small\ttfamily}}{}

% use upquote if available, for straight quotes in verbatim environments
\IfFileExists{upquote.sty}{\usepackage{upquote}}{}

% Use microtype if available
% must be loaded after fonts
\IfFileExists{microtype.sty}{%
\usepackage{microtype}
\UseMicrotypeSet[protrusion]{basicmath} % disable protrusion for tt fonts
}{}

% Footnote
% --------
\renewcommand{\footnoterule}{\noindent\smash{\rule[5pt]{1.0in}{0.1pt}}}

%\renewcommand*{\quotation}{\fontfamily{ptm}\selectfont\itshape}
%\renewcommand*{\quote}{\quotation}{\endquotation}
